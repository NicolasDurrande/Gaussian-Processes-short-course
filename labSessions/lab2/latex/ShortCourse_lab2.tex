\documentclass[12pt]{scrartcl}

\usepackage[utf8]{inputenc}

\usepackage{mathpazo} % math & rm
% \linespread{1.05}        % Palatino needs more leading (space between lines)
\usepackage[scaled]{helvet} % ss
\usepackage{courier} % tt
\normalfont
\usepackage[T1]{fontenc}

\usepackage{amsthm,amssymb,amsbsy,amsmath,amsfonts,amssymb,amscd}
\usepackage{dsfont}
\usepackage{tasks}
\usepackage{enumitem}
\usepackage[top=2cm, bottom=3cm, left=3cm , right=3cm]{geometry}
\usepackage{tikz}
\usetikzlibrary{automata,arrows,positioning,calc}


\begin{document}
\begin{center}
	\rule{\textwidth}{1pt}
	\\ \ \\
	{\LARGE \textbf{Lab 2 -- Design of Experiments}}\\ 
	\vspace{3mm}
	{\large Short course on Statistical modelling for optimization\\ \vspace{3mm}}
	{\normalsize N. Durrande - J.C. Croix, Universidad Tecnol\'ogica de Pereira, 2017}\\ 
	\vspace{3mm}
	\rule{\textwidth}{1pt}
	\vspace{5mm}
\end{center}
The first half of this lab session is dedicated to designing the best possible DoE. In the second one, we will run the actual experiments. 

%%%%%%%%%%%%%%%%%%%%%%%%%%%%%%%%%%%%%%%%%%%%%%%%%
\section{Design of experiments}
The aim of this section is to define a DoE of 40 points over $(0,1)^4$ that shows good space filling properties and good projection properties. Some useful functions (such as \texttt{discrepancy}, \texttt{minimax}, \texttt{maximin}, \texttt{IMSE}) are provided in the file \texttt{lab2.py}. The file \texttt{SobolSequence.py} will also be of particular interest to generate low discrepancy sequences. 

% \subsection*{Questions}
\paragraph{Q1.} Write a function that implements a Latin Hypercube Design. The function should take as parameters the number of points $n$ and the dimension $d$. It should return a np.array with shape $(n,d)$.

\paragraph{Q2.} Write a function that returns a Centroidal Voronoi Tesselation. You may use a k-means or a McQueen algorithm. The inputs and outputs should be as in Q1.

\paragraph{Q3.} Generate various DoE using the functions you just wrote (you may also consider \texttt{SobolSequence}). 

\paragraph{Q4.} Choose your favourite DoE using the various space filling criteria. Do not forget to test the projections on some variables. Justify your choice in the report. 

%%%%%%%%%%%%%%%%%%%%%%%%%%%%%%%%%%%%%%%%%%%%%%%%%
\section{Running the experiments}

\paragraph{Step 1.} Convert your DoE on $(0,1)^4$ to the hyper-rectangle you defined during the last lab session.

\paragraph{Step 2.} Save your design on a file.

\paragraph{Step 3.} Run the experiments on the simulator.

\end{document}
